\documentclass[en,cm,10pt]{inst}
\begin{document}
	\section*{Study Plan for Degree Project}
	Degree Project: NUMM03 for Master's Degree in Numerical Analysis \\
	Student: John Doe, XXXXXX-XXXX\\
	Advisor: --\\
	Examiner: --\\ 
	Time:  January 20, 2025 -- June 8, 2025\\
    Preliminary examination date: June 8, 2025
	
	\subsubsection*{Content}
	This project focuses on reduced-order modeling (ROM), a technique for simplifying complex systems, and its integration with novel machine learning approaches, specifically addressing model problems in numerical analysis and computational physics (e.g., Burgers' equation).\\
    
    \noindent The thesis integrates Operator Inference (OpInf), a data-driven method for approximating reduced-order operators, with adjoint methods and potentially automatic differentiation. OpInf uses time series data $\{a(t_i)\}$ from complex systems to infer reduced-order operators $\{\hat{f}(a, \theta)\}$. By optimizing parameters $\theta$ using a loss function tied to system dynamics, this work explores efficient gradient computation during training through adjoint or automatic differentiation.\\

    The main objectives of the thesis are:
    \begin{itemize}
        \item Derive and reformulate the adjoint problem for Neural ODEs in the context of OpInf's integral form.
        \item Bridge classical numerical analysis techniques (e.g., ROM) with modern machine learning methods.
    \end{itemize}
	
	\subsubsection*{Literature}
	%for instance Borevich--Shafarevich, \textit{Number Theory}.
    \begin{itemize}
    
        \item Quarteroni, Alfio and Manzoni, Andrea and Negri, Federico. \textit{Reduced basis methods for partial differential equations: an introduction}. 2015
        \item Peherstorfer, Benjamin and Willcox, Karen. \textit{Data-driven operator inference for nonintrusive projection-based model reduction}. 2016
        \item Ghattas, Omar and Willcox, Karen. \textit{Learning physics-based models from data: perspectives from inverse problems and model reduction}. 2021
        \item McQuarrie, Shane and Guo, Mengwu and Willcox, Karen. \textit{Bayesian operator inference for data-driven reduced-order modeling}. 2022
        \item Chen, Ricky and Rubanova, Yulia and Bettencourt, Jesse and Duvenaud, David. \textit{Neural ordinary differential equations}. 2018
        \item Antil, Harbir and Leykekhman, Dmitriy. \textit{A brief introduction to PDE-constrained optimization}. 2018
        \item Bradley, Andrew. \textit{PDE-constrained optimization and the adjoint method}. 2021
        
    \end{itemize}

    \newpage
	
	\subsubsection*{Time table}
	\begin{itemize}
        \item[-] Fill in agreement with the supervisor.
        \item[-] 15 January - 10 March: Reading the theory. Learning of the typing system LaTeX 
        \item[-] 11 March - 30 April: Execution. Most of the report writing.
        \item[-] 1 May - 31 May: Completion and closing. In this phase, the report is completed and proofread together with the advisor. The work is submitted to an examiner, to be presented at a seminar two weeks later.
    \end{itemize}
	\vspace{0.5cm}
    
	\noindent This study plan has been signed in two copies by the student and the advisor.
	
	\vspace{0.5cm}
	\noindent
	%Lund, January 20, 2025
	Lund, \today
	
	\vspace{1.0cm}
	\noindent Student \hspace{5cm}Advisor
	
	%%% SIGNATURE %%%
	\begin{tikzpicture}
		\node[above=1.5 cm,right=1.0cm] () {};
		\includegraphics[width=4cm]{signature.png};
	\end{tikzpicture}
	
\end{document}